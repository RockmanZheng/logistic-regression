\documentclass[a4paper,12pt]{article}
\usepackage{amsmath}
\usepackage{amsfonts}
\begin{document}

The foundations of the rigorous study of \emph{analysis}
were laid in the nineteenth century, notably by the
mathematicians Cauchy and Weierstrass. Central to the
study of this subject are the formal definitions of
\emph{limits} and \emph{continuity}.

Let $D$ be a subset of $\bf R$ and let
$f \colon D \to \mathbf{R}$ be a real-valued function on
$D$. The function $f$ is said to be \emph{continuous} on
$D$ if, for all $\epsilon > 0$ and for all $x \in D$,
there exists some $\delta > 0$ (which may depend on $x$)
such that if $y \in D$ satisfies
\[ |y - x| < \delta \]
then
\[ |f(y) - f(x)| < \epsilon. \]

One may readily verify that if $f$ and $g$ are continuous
functions on $D$ then the functions $f+g$, $f-g$ and
$f.g$ are continuous. If in addition $g$ is everywhere
non-zero then $f/g$ is continuous.

$$
\delta(i,j)=
\begin{cases}
1, & \quad \text{if } i=j\\
0, & \quad \text{otherwise}\\
\end{cases}
$$

$$\textbf{P}(\textbf{X}_i,u)\geq0, \quad i=1,\cdots,m,u=1,\cdots,k$$
$$\sum_{u=1}^k\textbf{P}(\textbf{X}_i,u)=1,\quad i=1,\cdots,m$$
$$\textbf{P}(\textbf{X}_i,Y_i)$$
%$$\textbf{P}(\textbf{X}_i,1)=\frac{e^{\textbf{\lambda} \textbf{X}_i}}{1+e^{\textbf{\lambda} \textbf{X}_i}}$$
$$\textbf{P}(\textbf{X}_i,1)=\frac{e^{\boldsymbol\lambda^T \textbf{X}_i}}{1+e^{\boldsymbol\lambda^T \textbf{X}_i}}$$
$$\textbf{P}(\textbf{X}_i,2)=1-\textbf{P}(\textbf{X}_i,1)$$
$$\textbf{X}=\left(\textbf{X}_1,\cdots,\textbf{X}_m\right)$$
$$\textbf{X}_i=\left[X_i^{(1)},\cdots,X_i^{(n)}\right]$$
$$X_i^{(j)}$$
$$m,\quad n,\quad k$$
$$\{1,\cdots,k\}$$
$$\textbf{Y}=(Y_1,\cdots,Y_m)$$
$$Y_i=u$$
$$k=2$$
$$\textbf{P}:\mathbb{R}^n\to\mathbb{R}^k$$
$$P(x\in1)P(x\in1,y\in1)=P(x\in1)P(y\in1)$$
$$P(x\in1,y\in1)=P(x\in1)P(y\in1)$$
$$\boldsymbol\lambda=[\lambda_1,\cdots,\lambda_n]$$
$$\textbf{P}(\textbf{X}_i,u)=\frac{e^{\boldsymbol\lambda_u^T\textbf{X}_i}}{\sum_{v=1}^k e^{\boldsymbol\lambda_v^T\textbf{X}_i}},\quad u=1,\cdots,k$$
$$\boldsymbol\lambda=(\boldsymbol\lambda_1,\cdots,\boldsymbol\lambda_k)$$
$$\boldsymbol\lambda_u=\left[\lambda_u^{(1)},\cdots,\lambda_u^{(n)}\right],\quad u=1,\cdots,k$$
$$\boldsymbol\lambda_2=\textbf{0}$$
$$\boldsymbol\lambda_v^T\textbf{X}_i-\boldsymbol\lambda_u^T\textbf{X}_i$$
$$\boldsymbol\lambda_v^T\textbf{X}_i>\boldsymbol\lambda_u^T\textbf{X}_i$$
$$\boldsymbol\lambda_v^T\textbf{X}_i<\boldsymbol\lambda_u^T\textbf{X}_i$$
$$\boldsymbol\lambda_v^T\textbf{X}_i\approx\boldsymbol\lambda_u^T\textbf{X}_i$$
$$u\quad v$$
$$P(\textbf{X}_1\in u_1,\cdots,\textbf{X}_m\in u_m)=\prod_{i=1}^m\textbf{P}(\textbf{X}_i,u_i)$$
$$\prod_{i=1}^m\textbf{P}(\textbf{X}_i,Y_i)$$
$$A$$
$$f(\boldsymbol\lambda)=\sum_{i=1}^m \ln\textbf{P}(\textbf{X}_i,Y_i)$$
$$\lambda_u^{(j)}$$
$$\frac{\partial\textbf{P}(\textbf{X}_i,u)}{\partial\lambda_u^{(j)}}=X_i^{(j)}\textbf{P}(\textbf{X}_i,u)\left(1-\textbf{P}(\textbf{X}_i,u)\right)$$
$$\frac{\partial\textbf{P}(\textbf{X}_i,u)}{\partial\lambda_v^{(j)}}=-X_i^{(j)}\textbf{P}(\textbf{X}_i,u)\textbf{P}(\textbf{X}_i,v),\quad u\neq v$$
\begin{align}
\frac{\partial f(\boldsymbol\lambda)}{\partial\lambda_u^{(j)}}=
&\sum_{i=1}^m\frac{1}{\textbf{P}(\textbf{X}_i,Y_i)}\frac{\partial\textbf{P}(\textbf{X}_i,Y_i)}{\partial\lambda_u^{(j)}}\\=
&\sum_{i=1}^m\frac{\delta(u,Y_i)}{\textbf{P}(\textbf{X}_i,u)}X_i^{(j)}\textbf{P}(\textbf{X}_i,u)\left(1-\textbf{P}(\textbf{X}_i,u)\right)\\
&-\sum_{i=1}^m\frac{1-\delta(u,Y_i)}{\textbf{P}(\textbf{X}_i,Y_i)}X_i^{(j)}\textbf{P}(\textbf{X}_i,Y_i)\textbf{P}(\textbf{X}_i,u)\\=
&\sum_{i=1}^m X_i^{(j)}\left[\delta(u,Y_i)-\textbf{P}(\textbf{X}_i,u)\right]
\end{align}
$$\sum_{i=1}^m X_i^{(j)}\delta(u,Y_i)=\sum_{i=1}^m X_i^{(j)}\textbf{P}(\textbf{X}_i,u),\quad j=1,\cdots,n,u=1,\cdots,k$$
$$Y_{i'}=u$$
$$j$$
$$\lambda_u^{(j)}\gets\lambda_u^{(j)}+h\frac{\partial f(\boldsymbol\lambda)}{\partial\lambda_u^{(j)}}$$
$$\textbf{K}=[1,\cdots,k]$$
$$\boldsymbol\delta(\textbf{Y},\textbf{K})=\left(\delta(Y_i,u)\right)_{i=1,u=1}^{m,k}$$
$$\textbf{P}(\textbf{X},\textbf{K})=\left(\textbf{P}(\textbf{X}_i,u)\right)_{i=1,u=1}^{m,k}$$
$$\boldsymbol\lambda\gets\boldsymbol\lambda+h\textbf{X}\left[\boldsymbol\delta(\textbf{Y},\textbf{K})-\textbf{P}(\textbf{X},\textbf{K})\right]$$

$$\textbf{P}(\textbf{X}_i,u)=\frac{1}{\sum_{v=1}^k e^{\left(\boldsymbol\lambda_v-\boldsymbol\lambda_u\right)^T\textbf{X}_i}},\quad u=1,\cdots,k$$

$$i\quad j\quad \mathbb{R}^{n\times k}$$

\begin{align}
\frac{\partial^2 f(\boldsymbol\lambda)}{\partial\left[\lambda_u^{(j)}\right]^2}=
&-\sum_{i=1}^m X_i^{(j)}\frac{\partial\textbf{P}(\textbf{X}_i,u)}{\partial\lambda_u^{(j)}}\\=
&-\sum_{i=1}^m\left[X_i^{(j)}\right]^2\textbf{P}(\textbf{X}_i,u)(1-\textbf{P}(\textbf{X}_i,u))
\end{align}

$$H(\textbf{P}) = -\sum_{u=1}^k\sum_{i=1}^m\textbf{P}(\textbf{X}_i,u)\ln\textbf{P}(\textbf{X}_i,u)$$

$$\textbf{X}^{(j)}\neq\textbf{0}$$

$$\sum_{i=1}^m \textbf{X}_i\delta(u,Y_i)=\sum_{i=1}^m\textbf{X}_i\textbf{P}(\textbf{X}_i,u),\quad u=1,\cdots,k$$

$$\beta_i$$

$$L=\sum_{i=1}^m\sum_{u=1}^k\left[\boldsymbol\lambda_u^T\textbf{X}_i\textbf{P}(\textbf{X}_i,u)+\beta_i\textbf{P}(\textbf{X}_i,u)-\textbf{P}(\textbf{X}_i,u)\ln\textbf{P}(\textbf{X}_i,u)\right]$$

$$\frac{\partial L}{\partial\textbf{P}(\textbf{X}_i,u)}=\boldsymbol\lambda_u^T\textbf{X}_i+\beta_i-\ln\textbf{P}(\textbf{X}_i,u)-1$$

$$\textbf{P}(\textbf{X}_i,u)=e^{\beta_i-1}e^{\boldsymbol\lambda_u^T\textbf{X}_i}$$

$$e^{\beta_i-1}=\frac{1}{\sum_{u=1}^ke^{\boldsymbol\lambda_u^T\textbf{X}_i}}$$
\end{document}
